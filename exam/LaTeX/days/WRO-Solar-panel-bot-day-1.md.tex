\textbf{Date:} May 19th 2014\\\textbf{Duration:} 9-16\\\textbf{Group
members:} Henrik, Jakob, Jesper

\subsection{Goals for today}

First thing we will do is work out a threat analysis for the World Robot
Olympiad 2014 challenge.

\subsection{Threat analysis}

\begin{itemize}
\itemsep1pt\parskip0pt\parsep0pt
\item
  Positioning, how will the system know where it is
\item
  Strategy for turning solar panels
\item
  Strategy for replacing solar panels
\item
  The system must be able to pass solar panels without moving them
\item
  How will we pick up a solar panel to move it around
\item
  More importantly, how will we transport a solar panel back between the
  correctly positioned ones without touching them?
\item
  Keeping track of points on the robot
\item
  Error handling, what will the robot do if it loses track of its
  position? Important to not keep going and possibly lose points
\item
  Detecting state of a solar panel; the color sensor returns black if
  nothing is seen, how will we distinguish black from nothing?
\item
  Motor limitation, we can only have three motors
\item
  Handling, can the robot turn around without issue?
\end{itemize}

In the official challenge, there is a bridge on the track. This is not
present on our test track, and so we will disregard it in our project.

\subsection{Navigation}

\subsubsection{Goal}

Our main focus today is to come up with a strategy for how to navigate
the track.

\subsubsection{Plan}

\begin{itemize}
\itemsep1pt\parskip0pt\parsep0pt
\item
  Brainstorm ideas for how to navigate the track.
\item
  Discuss suggestions with their pros and cons.
\item
  Decide on a method to use in our final solution.
\end{itemize}

\subsubsection{Possible solutions}

\begin{itemize}
\itemsep1pt\parskip0pt\parsep0pt
\item
  Tacho counter (Cartesian coordinate mapping).
\item
  GPS + compass.
\item
  Line follower.
\end{itemize}

\subsubsection{GPS}

We could use an ordinary GPS sensor to find the system's position and a
compass sensor to find its orientation. This is a very robust solution,
as the system can, at any time it becomes disoriented, just stop and
read its exact position, and it is independent of lines to follow, tacho
counter etc. It is possible to get an accuracy of about 2 cm using L1
GPS {[}1{]}.\\This accurate GPS module takes 12 and a half minute to
find the position, and as we have a strict 2 minute time limit, using
this method for navigation is infeasible. Alternatively we could use
conventional GPS modules, but these are only accurate down to 3 meters,
and as that is about the size of the track we are driving on, it will
not do us any good. In addition to this, the compass sensor will be
useless if there is any electromagnetic interference nearby. Also, the
track is inside a building, which heavily limits the effectiveness of a
GPS module.

\subsubsection{Tacho counter}

We could use the tacho counter to keep track of how far we have moved
and use this knowledge to keep track of our position in a Cartesian
coordinate system. We can also find our orientation from the tacho
counter. The motors are slightly inaccurate however, and after driving
around for long enough, our position will start to become inaccurate
{[}2{]}. And there is no way to correct this inaccuracy, so this
solution is not very error resilient.

\subsubsection{Line following}

The whole track has black lines that we can follow using a light sensor,
and we can find our position using certain points and crossings on the
track.\\Issues we might run into using this method is calibrating the
light sensor to properly recognize black, white, green and gray, as we
will not know the exact lighting conditions at all times. Another issue
is positioning the light sensor, as we usually position right about
where the solar panels will be on the track, and we cannot move these or
we will have points deducted.

\subsubsection{Conclusion}

Out of these three proposals, line following is the only one that
appears feasible, and so we decide to continue with this approach.

\subsection{Positioning light sensor}

We have a couple of different ways to place the light sensors. - We can
place it near the surface of the track at an angle, so the light sensor
can be far enough to the side to avoid hitting the solar panels. - We
can have the light sensor point directly down, but placed in a height
that allows a solar panel to pass underneath.

We will do some measurements to see which solution is better at
distinguishing black and white. To do these measurements, we have
written a program to save readings for easier comparison.

\subsubsection{Results}

First result for each color is with the red light on, second is without

  \begin{longtable}[c]{@{}lcrrr@{}}
    \hline\noalign{\medskip} Color & 3 cm - normalized & 3 cm - raw &
    tilt - normalized & tilt - raw \\\noalign{\medskip}
    \hline\noalign{\medskip} Green & 44 & 459 & 50 & 517
    \\\noalign{\medskip} Green & 44 & 458 & 50 & 515
    \\\noalign{\medskip} Black & 47 & 488 & 53 & 544
    \\\noalign{\medskip} Black & 47 & 483 & 53 & 548
    \\\noalign{\medskip} White & 53 & 544 & 56 & 583
    \\\noalign{\medskip} White & 52 & 541 & 56 & 581
    \\\noalign{\medskip} Grey & 48 & 495 & 51 & 523
    \\\noalign{\medskip} Grey & 47 & 486 & 50 & 514
    \\\noalign{\medskip} \hline
  \caption{Results for each color in the map (with light sensor)}
  \end{longtable}

\subsubsection{Conclusion}

As seen from the values, both approaches are viable solutions. From the
datalogger output, we see that the tilted approach yields more accurate
results, but we are worried that the sensor might bump into the panels.
The lifted approach is also simpler to construct, and so we decided to
go with that.

\subsection{Color sensor, checking current state of a solar panel}

\subsubsection{Goal}

Get the system to distinguish between a solar panel in the correct
position, wrong position, a defect solar panel, or no solar panel at
all.

\subsubsection{Plan}

The way we are going to test if the system can distinguish between the
states of the solar panels is by holding the color sensor in the
position we expect it to be in in the final design and reading values on
different solar panels.\\And to clarify, we expect the color sensor to
be placed underneath the robot, a bit off to one side. This will allow a
solar panel to pass by without bumping into the sensor and thus
subtracting points from the final score.

\subsubsection{Results}

We see that by placing the color sensor as described above and reading
values for the four different states, we can easily distinguish between
the states. Thus, this is a good way to position the color sensor, and
it will allow us to distinguish between states.

\begin{longtable}[c]{@{}lcrr@{}}
  \hline\noalign{\medskip} State (color) & Red & Green & Blue
  \\\noalign{\medskip} \hline\noalign{\medskip} Working (blue) & 527
  & 536 & 576 \\\noalign{\medskip} Turned (red) & 647 & 581 & 583
  \\\noalign{\medskip} Broken (black) & 522 & 515 & 518
  \\\noalign{\medskip} Empty (none) & 742 & 741 & 741
  \\\noalign{\medskip} \hline
\caption{Results for solar pannels different state}
\end{longtable}

\subsection{Conclusion}

\begin{itemize}
\itemsep1pt\parskip0pt\parsep0pt
\item
  We decided on a way to navigate the track.
\item
  We found a good way to position the light sensor so that the vehicle
  can follow a line.
\item
  We found a good way to position the color sensor so that we can
  distinguish between working, defect, missing and turned solar panels.
\end{itemize}

\subsection{References}

{[}1{]} J. A. Farrell, T. D. Givargis, and M. J. Barth, ``Real-Time
Differential Carrier Phase GPS-Aided INS,'' in '' IEEE Trans. Contr.
Syst. Technol., vol.~8, no. 4 pp.~709--721, July 2000. 1991. CA, Jan.
1998, pp.~345--354.

{[}2{]} Lesson 9 lab report
