\textbf{Date:} June 18th 2014\\\textbf{Duration:}
9:30-13:15\\\textbf{Group members:} Henrik

\subsubsection{Goals}

\begin{itemize}
\itemsep1pt\parskip0pt\parsep0pt
\item
  Calibration
\item
  Make everything work together, i.e.~make the strategy for driving
  around the track, the strategy for making the robot turn solar panels
  and the strategy for exchanging broken solar panels work together.
\end{itemize}

\subsubsection{Calibration}

As the robot is very sensitive to the lighting conditions of the room,
we have to change the calibration method again as we cannot otherwise
guarantee that the robot can finish a complete tour if the lighting
conditions change (The sun appears, people are casting shadows over the
table, a cloud blocks out the sun, etc.). The first plan is to mount a
light sensor on the top of the robot pointing upwards, which will only
be used to measure the ambient light, both while calibrating but also
dynamically adjusting the values we use for the other sensor on the run.

We have noticed when the sun is brighter, there are also darker shadows,
so the robot is more affected in some directions than others by this. In
order to solve this problem, we either have to shield the robot from the
sun completely or make a function that correlates the darkness of the
shadows in the room with the amount of light in the room.

\textbf{Results}\\After having implemented this and made a few test runs
where the light has suddenly fluctuated, it seems it is a reasonable way
to fix the problem; there are still a few big deviations from the line,
but this seems to be because the sensors measure the light differently
(Either because of differences internally in the sensor or because of
the location of the sensors).

\subsubsection{Strategy}

As we explained yester, the strategy for driving is first to drive to a
row, fix the row as explained yesterday and then drive to the next row,
etc.

As we started implementing this and testing it, we discovered that the
function to replace broken solar cells still have some issues, so we are
focusing on the inactive solar cells only for now. There were quite a
few issues with the robot when it had to turn the solar panels. We
tweaked the parameters back and forth, but nothing helped. Finally, the
battery ran dry, and when we replaced it with a new one, the robot
functioned without major issues when driving through the grid with all
solar panels already active.
