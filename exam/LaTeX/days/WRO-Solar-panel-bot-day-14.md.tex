\textbf{Date:} June 17th 2014\\\textbf{Duration:}
9:30-22:15\\\textbf{Group members:} Henrik

\subsection{Goals}

\begin{itemize}
\itemsep1pt\parskip0pt\parsep0pt
\item
  Work out a new way of calibrating the sensors
\item
  Work on the code and make it more dynamic
\item
  Write code to make the robot do correct actions for an entire row of
  solar panels
\end{itemize}

\subsection{Calibration}

Yesterday we had some problems with the calibration; we have only been
calibrating in two directions so far (East and west with regards to our
previously defined cardinal directions), so when the robot had to follow
a line very accurately in order to hit a solar panel correctly, the
calibrated offset value was obtained from an angle that wasn't entirely
correct, which means the robot drove in a wrong distance from the line.

Today we have changed the robot to calibrate in all 4 cardinal
directions in the hope that this will improve results. When we tested
it, it turned out it did indeed improve results, but later on in the day
we discovered that even just a small cloud blocking the natural lighting
affected our robot drastically. We have not been able to find a solution
to this problem unless we find a room that only has artificial lighting.

\subsection{The code}

The code is nearing completion, although there is still missing an
overall structure that implements an overall strategy for the behaviour
of the robot. We have started on this and decided that our first
strategy shall work as follows:
\begin{enumerate}
\item Drive to a row of solar
panels.
\item Pass over the row and note whether the solar panels
are active or inactive; if they are broken, they will be replaced
immediately. This means the robot will pick it up immediately, turn
around and drive back to the warehouse, fetch a new solar panel, drive
back to the row, find the correct spot, place the solar cell and mark it
as active in memory and then continue passing over the row.
\item
When the state of all three solar panels have been noted, the strategy
continues as below:
\begin{itemize} 
\item If solar panel 3 is inactive, it will be rotated,
such that the robot is facing south, which makes it easy for the robot
to drive to spot \#1 or \#2 and pick up a solar panel there.
\begin{itemize} 
\item If solar
panel 1 is inactive, the robot will drive to this one and rotate it, so
that the robot is once again facing north, ready to rotate solar panel 2
if needed. If this is not needed, the robot drives backwards out of the row
using tacho count.
\item If solar panel 1 is not inactive, the robot will drive to solar panel 2 and rotate it, after which it will drive backwards out of the row using tacho count.
\end{itemize}
\item If solar panel 3 is not inactive, the robot will drive
backwards to solar panel 2.
\begin{itemize} 
\item If this one is inactive, it will be rotated
and the robot will be facing south, ready to turn solar panel 1 if
needed. If this is not needed, it simply drives out of the row. 
\item If solar panel 2 was not inactive, the robot drives backwards to solar panel 1 and rotates it if needed. Otherwise, it simply drives backwards out of the row using tacho count.
\end{itemize}
\end{itemize}
The general idea is to try and minimize the number of times the robot
has to drive backwards after rotating a solar panel, which is what is
hardest for the robot at the moment.
\item When a row is completed (All solar panels are active), the
robot drives to the next row and the process is repeated.
\end{enumerate}
We have also made the code that makes it possible for the robot to
replace one broken solar panel and return to the right spot afterwards;
there are still a few bugs here, but it almost works.
