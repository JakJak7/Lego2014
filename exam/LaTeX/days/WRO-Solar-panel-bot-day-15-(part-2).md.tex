\textbf{Date:} June 18th 2014\\\textbf{Duration:} 13-02\\\textbf{Group
members:} Henrik, Jesper, Jakob

\subsection{Today}

More test runs, it does alright although it often fails to recognize the
black solar panels.\\We tried cleaning the track in hopes of improving
line following, as the track is incredibly dirty at this
point.
\begin{figure}[hbt]
  \centering
  \includegraphics[scale=0.13]{../experiments/images/cleaning.jpg}
  \caption{Picture of the track, on the right side it has been cleaned, and og the left it has not.}
\end{figure}
We now have the vehicle display the RGB and light values, so that we can
read them and hard code them. This way, the vehicle won't have to
shuffle around at the beginning in order to get readings, and we'll save
some time on the full run.

We saw another group made a paper armor for the vehicle, shielding from
light. This is to avoid any changes in the ambient light, and we decided
to try this approach on our own.\\This didn't help, and actually made
our vehicle run worse, so we got rid of it.

Next up we tried moving the entire track to another location with more
stable lighting conditions. Here we had the option to turn the lights on
or off. Turning the lights off didn't help, because the color sensor
registered the floodlight from our light sensor as the color red, and
turning off the floodlight isn't an option in the dark.\\With the lights
on we achieved much better results.\\Now we can get the vehicle to turn
all panels, and replace the broken ones. As the vehicle was running low
on battery, we switched to a fresh battery, and now it would no longer
turn nor back up correctly. Battery charge is something we have to keep
in mind, now that we've fixed the lighting conditions.\\Another thing
we've yet to implement, is have the vehicle pick up replacement panels
from the three slots in storage. So far we've been taking them all from
the first slot, and inserted a new one when necessary. This is illegal
in the final competition, so we'll have to make our vehicle obey the
rules.

\subsection{Conclusion}

We have moved the whole track to a room with fixed lighting conditions,
and thus (hopefully) eliminated the calibration issues we have been
troubled with for a while. Currently our vehicle runs great here. Next
up we'll get our vehicle to be able to pick up fresh solar panels from
storage, that is if we can make it in time, as tomorrow is the day of
the presentation.
