\textbf{Date:} June 16th 2014\\\textbf{Duration:} 14-19\\\textbf{Group
members:} Henrik

\subsection{Recap \& Goals}

It has been a week since we last worked on the project due to exams. So
far we have made the robot find its way to grid \#1 and turn all solar
panels on this grid. We have also made the robot find its way to grid
\#1, spot \#2, carry a broken solar panel back to the warehouse from
this position and return with a working solar panel. (Grid \#1 is the
east-most line of solar panels according to our earlier cardinal
directions definition)

Today the plan is to see if things still work and then continue
refactoring the code, such that the aforementioned functions (Turning
and replacing solar cells) continue to work while changing the code from
being hard coded to being more dynamic, so that the robot will do the
right things no matter how the track has been set up with regards to
solar panels.

\subsubsection{Do things still work?}

We have run our old program, and aside from a few bugs due to the
lighting, it worked out. We may need to find a better way to measure the
light, but that might mean a lot of analysis on different data, which we
might not be able to do in time unless we have an additional month.

\subsection{Refactoring}

The code has been split into more functions and classes, so that we can
more easily call the different commands (Turn robot, move robot, turn
solar panel, etc.). We started out by building a program with this new
data structure to see if things were still working; the program drove
out to grid \#1, spot \#1 and turned the first solar panel. It worked
just as well as it did with the old code.

We then decided to try out a theory which we have been considering for
the past week. The theory was to use a light sensor on the backside of
the robot to find the spot that the robot's rotation would revolve
around. As the light sensor is the positioned on the middle of the axle
between the two motors that rotate the robot, the robot will, when the
backside light sensor is above an intersection, after rotating be
positioned so it is correctly aligned immediately. As alignment has been
a big problem earlier, this would be a very nice improvement. We
switched out the P-controller with a new function for terminating the
movement of the robot; this new function terminates the movement when
the backside sensor hits an intersection, at which point we start
rotating. There were a few problems in the initial runs as, after
rotating, the backside sensor would still be positioned in the middle of
a line, causing the robot movement to terminate again. After a lot of
tests and attempts at getting the backside sensor to not make the robot
terminate again, we dropped the idea again as we could not get it to
function correctly.

We then wanted to get the robot to run faster by doing faster turns and
driving faster on the long straight lines. This gave us some problems as
our turn method is based on a tacho counter; as we increased the speed
from 40 to 70 (Out of a scale of 0 to 100), the robot turned about 30
degrees too much. However, at the straight lines, it worked fine, so we
are still able to lower the total run time of the robot by increasing
the speed at these areas and then lowering it again for the turns and
solar panel rotations.

We should be able to implement a new method for turning the solar panel
which can turn it faster, but it will only be in connecting with
refactoring the code once the first program for completing the track has
been finished.

\subsection{Conclusion}

Today we have refactored the code, so that it will be easier to continue
working with, tested (and rejected) a new way of turning the robot
(Using a backside sensor), tested code for approaching a solar panel and
rotating it as well as made attempts at improving the speed of the
robot, which worked in some places.

No major changes in the way the robot looks or works, so no pictures or
videos for today.
