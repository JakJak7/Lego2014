\textbf{Date:} June 9th 2014\\\textbf{Duration:} 9-20\\\textbf{Group
members:} Henrik, Jakob, Jesper

\subsection{Goals}

\begin{itemize}
\itemsep1pt\parskip0pt\parsep0pt
\item
  Get the robot to properly hold on to the solar panels
\item
  Get it to replace the broken ones
\end{itemize}

\subsection{Results}

Everything failed in our initial run because the sunlight was brighter
than usual, and so the robot was unable to find any black lines. We
changed the light sensor values to recalibrate it.\\We had some
inconsistent results turning the first solar panel, as the vehicle would
often bump into it and move it out of place. To counter this, we lowered
the speed of the vehicle as it approaches a solar panel, and the results
got much more consistent.\\So, now we can turn a solar panel, and we can
pick up a broken panel and carry it back to the start. Now we have a
problem; we have to exchange the broken panel for a new one, but we have
no obvious way of putting down the broken panel while picking up the
functional one.\\Immediate solution: place the broken panel somewhere
else, and go pick up the new one.

We did come up with a clever solution where we move the broken panel
into the storage area in front of the fresh, working one, then use our
mechanism to move the broken one aside and grab the new
one.\\\url{https://www.youtube.com/watch?v=C0-sibkTmmU}\\As seen in the
video, the vehicle is able to replace the solar panel and leave the
broken one in storage, which according to the rules is the proper way to
do it.

\subsection{Conclusion}

We got the vehicle calibrated for the lighting conditions, and got it to
replace broken solar panels. Next thing we should look at is getting it
to replace several panels in one run, as it currently is only capable of
taking the first fresh one.
