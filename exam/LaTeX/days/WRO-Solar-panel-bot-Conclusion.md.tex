\subsection{Conclusion}

In the end, we think our project was quite successful. We initially
feared that we would not be able to get anywhere near the 2 minute time
limit that the official WRO competition has, but we actually have a run
with a legal set up of solar panels that only takes 2 minutes and 19
seconds to complete, so we actually made it quite close.

During the project, we have had a lot of issues with the lighting
conditions in the rooms we were running our robot in. This is in part
due to our choice to try and limit the number of sensor we used; our
final robot ended up using only 2 sensors - 1 color sensor for detecting
the color of the solar panels and 1 light sensor to follow the lines on
the track. It was also partly caused by the large distance our light
sensor had to the ground, but achieving this large distance and still
being able to follow the lines is also something we are quite proud of.

\subsection{Future work}

The robot is able to complete the track for full points if there is only
one broken solar panel (Aside from the time limit). If there are
multiple broken solar panels, we are unable to handle more than the
first one of them. If we had more time, this is one of the first issues
we would be looking at.

We also got some other ideas late in the project for optimizing the runs
for the robot. One of these is based on the fact that an official WRO
setup will have 4-6 solar panels that are incorrectly configured. This
is decided before the run is started. As such, we could implement a
counter that increments every time we have reconfigured a solar panel
and once this counter matches the number of solar panels that needs to
be configured, we could simply return to home. In an ideal case, this
would allow us to save precious seconds by not having to check the third
row at all.

Another thing we did not have the time to implement was error handling.
If the robot malfunctions for any reason, it will continue to execute
the entire program instead of stopping. However, as the rules state that
the team is simply allowed to call ``Stop'' to stop the attempt and
begin scoring, handling errors by terminating the program is not
necessarily. Some form of error correction code could have been nice to
have the time to implement however. This could for example involve the 2
sensor ports that we still have open.
